\href{https://www.redblobgames.com/grids/hexagons/}{\tt Red Blob Games Website}で 説明されている立方体座標・\+Q\+R座標の\+C\+::での実装 ~\newline
This C\# code implements Cube-\/ and Q\+R-\/based coordinates described in \href{https://www.redblobgames.com/grids/hexagons/}{\tt Red Blob Games Website.}

\section*{Usage}

\subsection*{Coordinates}

There are 2 coordinates supported -\/ Cube and QR(Axial.) ~\newline
Constructors of those value objects are private, and you must make through specific methods, {\ttfamily from\+Cube} and {\ttfamily from\+QR} respectively.

The supported coordinates makes creating concentric hexagonal map easy. Better supports for map that spreads in the form of a rectaingle are planned through offset coordinates.

{\bfseries Note} that whether your hexagonal map is pointy-\/topped or flat-\/topped is completely up to you. These coordinates should work fine under both circumstances.

{\bfseries Also note} that you M\+U\+ST N\+OT directly change the value for the coordinates. You can\textquotesingle{}t. If you want to modify the value and save it somewhere else, just create new coordinate object with modified value.

\subsubsection*{Cube Coordinates}

\begin{quote}
Let\textquotesingle{}s take a cube grid and slice out a diagonal plane at {\ttfamily x + y + z = 0.} This is a {\itshape weird} idea but it helps us make hex grid algorithms simpler. In particular, we can reuse standard operations from cartesian coordinates\+: adding coordinates, subtracting coordinates, multiplying or dividing by a scalar, and distances. ~\newline
 -\/ Red Blob Games \end{quote}


Use {\ttfamily Hexagonal\+Map.\+Domain.\+Hex\+Map.\+Cube\+Coordinates} to represent the cube coordinates.


\begin{DoxyCode}
\{c#\}
// The center
var center = CubeCoordinates.fromCube(0,0,0);
// Top hexagon (in flat-top system)
var top = CubeCoordinates.fromCube(0,1,-1);
// You can access each coordinate via:
top.x // -> 0
top.y // -> 1
top.z // -> -1
// Note that in Cube coordinate, x + y + z is always 0. If they aren't, it will yield an exception.
var fail = CubeCoordinates.fromCube(1,2,3); // -> ArgumentException
\end{DoxyCode}


Addition and Deduction are implemented. This is important when finding the 6 neighbours.


\begin{DoxyCode}
\{c#\}
// CubeCoordinates has a method to return a list of relative positions
// of the 6 neighbours. 
var relativeNeighbours = CubeCoordinates.AdjacentRelatives();
var origin = CubeCoordinates.fromCube(0,0,0);
foreach (var relative in relativeNeighbours) \{
    origin + relative // -> is the coordinate for one of the neighbour of origin.
\}
\end{DoxyCode}


Find if two coordinate match simply using {\ttfamily ==} (or don\textquotesingle{}t match using {\ttfamily !=})


\begin{DoxyCode}
\{c#\}
var a = CubeCoordinates.fromXYZ(0,0,0);
var b = CubeCoordinates.fromXYZ(0,0,0);
var c = CubeCoordinates.fromXYZ(1,0,-1);

a == b // -> true
a != b // -> false
a == c // -> false
a != c // -> true
\end{DoxyCode}


\subsubsection*{QR Coordinates (Axial Coordinates)}

\begin{quote}
The axial coordinate system, sometimes called \end{quote}
“trapezoidal” or “oblique” or “skewed”, is built by taking two of the three coordinates from a cube coordinate system. ~\newline
-\/ Red Blob Games

This coordinate is extremely useful for drawing, since you can convert the hexagons\textquotesingle{} position with little math.

Use {\ttfamily Hexagonal\+Map.\+Domain.\+Hex\+Map.\+Q\+R\+Coordinates} to represent the QR coordinates. ~\newline
 
\begin{DoxyCode}
// The center
var center = QRCoordinates.fromQR(0,0);
// Top hexagon (in flat-top system)
var top = QRCoordinates.fromQR(0, -1);
\end{DoxyCode}


{\ttfamily Q\+R\+Coordinates} cannot give you the relative coordinate of 6 neighbours, but addition, deduction, equality checks are implemented. See Cube Coordinates for usage.

\subsection*{Converting Coordinates}

Cube Coordinates and QR Coordinates can be converted into each other. Simply call\+:


\begin{DoxyCode}
\{c#\}
var qr = QRCoordinates.fromQR(0,0);
var cube = CubeCoordinates.fromCube(0,0,0);
QRCoordinates.fromCube(cube); // -> returns a QR coordinate for the given Cube coordinate
CubeCoordinates.fromQR(qr); // -> returns a Cube coordinate for the given QR coordinate
qr.toCube(); // -> converts itself (QR) into Cube coordinate
cube.toQR(); // -> converts itself (Cube) into QR coordinate
\end{DoxyCode}


You can also directly create those coordinates with values for other system\+:


\begin{DoxyCode}
var qr = QRCoordinates.fromCube(0,0,0); // -> is a QR coordinate that is equal to Cube coordinate (0,0,0)
var cube = CubeCoordinates.fromQR(0,0); // -> is a Cube coordinate that is equal to QR coordinate (0,0)
\end{DoxyCode}


\subsection*{Field and Cells}

T\+O\+DO 